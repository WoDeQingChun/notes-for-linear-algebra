\chapter{向量空间}
\section{预备知识-域}
运算说白了就由运算对象按照一定的运算法则生成运算结果。这样就很容易用映射来给出运算的数学定义
\begin{definition}
    对于非空集合$X,Y$,我们称映射$\varphi : X\rightarrow Y $为
    从$X$到$Y$的一元运算,当$X=Y$时,称$\varphi$是定义在$X$上的一元(代数)运算
\end{definition}
\begin{definition}
    对于非空集合$X,Y,Z$,我们称映射$\varphi : X \times Y \rightarrow Z $为
    从$X\times Y$到$Z$的二元运算,当$X=Y=Z$时,称$\varphi$是定义在$X$上的二元(代数)运算
\end{definition}

类似地可定义$n$元代数运算

强调几个点:
\begin{itemize}
    \item $\varphi$只是一个抽象的运算符号,可以是任何东西,但数学中常用$+,\cdot, * $等表示二元运算符
    \item 运算符的位置有前缀、中缀、后缀三种。常用的是中缀,例如将$\circ(x,y)$写成$\circ x y,x \circ y, x y \circ$
    \item 对于定义在非空集合$X$上的一个运算$*$,其封闭性显然已蕴含在定义中
\end{itemize}

在集合之上定义了运算之后,这种运算就赋予了集合元素之间一种代数结构,
例如在$\mathbb{N}$之上定义了加法之后,就有$1+3=4$,
这就在这三个元素之间形成了结构
\begin{definition}
    设$*$是定义在非空集合$S$上的一个运算,则称二元组$(S,*)$
    为一个(有一个代数运算的)代数系
\end{definition}

类似的,可以定义含更多个运算的代数系

对于含一个二元代数运算的代数系,我们关注该运算的交换律和结合律
\begin{definition}[交换律、结合律]
    设$(X,*)$是一个代数系,$*$是二元运算
    \begin{itemize}
        \item 若$\forall a,b \in X$,恒有
        $a*b=b*a$,则称$*$满足结合律
        \item 若$\forall a,b,c \in X$,恒有
        $(a*b)*c=a*(b*c)$,则称$*$满足结合律
    \end{itemize}
\end{definition}

同样的,二元代数运算的单位元素,以及由此引入的逆元素的概念同样很重要
\begin{definition}[单位元素、逆元素]
    设$(S,*)$是一个代数系,$*$是二元运算
    \begin{itemize}
        \item 若$\exists e \in S$,使得$\forall a \in S$,恒成立$e*a=a*e=a$,
        则称$e$为$*$的单位元素(也叫幺元,其中幺有数目中的一的含义)。类似地可以
        定义左单位元素和右单位元素的概念。
        \item 若$\forall a \in S,\exists b \in S$,使得$a*b=b*a=e$
        则称$b$是$a$在运算$e$下的逆元。
    \end{itemize}
\end{definition}

当一个代数系有两个二元代数运算时,这两个运算的交互能否满足分配律是我们关注的
\begin{definition}
    设$(S,*,+)$是一个代数系,$*,+$是二元运算,若$\forall a,b,c \in S$,恒有
    $a*(b+c)=a*b+a*c$,则称$*$对$+$满足左分配律,类似地可以定义右分配律,左右分配律都满足则称$*$对$+$满足分配律
\end{definition}

有了上面的准备,我们可以着手定义\textbf{域}
\begin{definition}
    设$(S,+,\cdot)$是一个代数系统,$+,\cdot$是二元运算(不妨分别称之为加法和乘法),
    则$(S,+,\cdot)$是一个域当且仅当满足以下五个条件:
    \begin{enumerate}
        \item $+,\cdot$满足交换律
        \item $+,\cdot$满足结合律
        \item $+,\cdot$有单位元(不妨分别记作0,1)
        \item $\forall x \in S$, 存在加法逆元;
                $x\neq 0$时,存在乘法逆元
        \item $\cdot$对$+$有分配律
    \end{enumerate}
\end{definition}

进一步地,不妨将$a$的加法逆元记为$-a$,乘法逆元记为$a^{-1}$,将减法$-$和除法$\div$分别定义为
\[a-b=a+(-b);a \div b = a \cdot b^{-1}\]
\begin{theorem}
    由$a$是$-a$的加法逆元,是$a^{-1}$的乘法逆元立即可得
    \begin{itemize}
        \item $a=-(-a)$
        \item $a=(a^{-1})^{-1}$
    \end{itemize}
\end{theorem}
\begin{theorem}[消去律]
    $(F,+,\cdot)$是一个域,$\forall a,b \in F$,有
    \begin{itemize}
        \item 若$a+b=a+c$,则$b=c$
        \item 若$a \cdot b=a \cdot c$且$a \neq 0$,则$b=c$
    \end{itemize}
\end{theorem}
\begin{corollary}
    域中的单位元、逆元都唯一
\end{corollary}
\begin{theorem}
    $(F,+,\cdot)$是一个域,$\forall a,b \in F$,有
    \begin{itemize}
        \item $a \cdot 0 = 0$
        \item $(-a) \cdot b = a \cdot (-b) = -(a \cdot b)$
        \item $(-a) \cdot (-b) = a \cdot b$
    \end{itemize}
\end{theorem}
\begin{corollary}
    域中的加法单位元没有乘法逆元
\end{corollary}