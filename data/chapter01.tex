\chapter{线性方程组}
线性方程组是高等代数研究问题的起点
\section{一些概念}
由对$n$元线性方程组的高斯消元法求解引入矩阵及其相关基本概念
\begin{equation*}
    \begin{cases}
        a_{11}x_1+\cdots+a_{1n}x_n=b_{1}\\
        a_{21}x_1+\cdots+a_{2n}x_n=b_{2}\\
        \cdots\\
        a_{s1}x_1+\dots+a_{sn}x_n=b_{s}
    \end{cases}
\end{equation*}

求解过程中不难发现方程组的解只与系数和常数项有关。为此我们从中抽象出一个数学模型,把它叫矩阵。
\begin{definition}
    由$s\cdot n$个数排成$s$行,$n$列的一张数表称为一个$s\times n$矩阵,矩阵常用大写字母$A,B,C$表示.若令$A=(a_{ij})_{s\times n}$,
    则其第i行,第j列个元素可表示为$A(i;j)$或$a_{ij}$
\end{definition}
\begin{definition}
    线性方程组系数所组成的矩阵称为系数矩阵,带上常数项后称为增广矩阵
\end{definition}

易知,高斯消元法中所用到的变换不改变方程组的解,因此引入矩阵初等行变换的概念
\begin{definition}
    称以下三种变换为矩阵的初等行变换:
    \begin{itemize}
        \item 用一个非零数乘以一行
        \item 交换两行
        \item 用一行的倍数加到另一行
    \end{itemize}
\end{definition}

由最后的求解又引入了行阶梯型矩阵和行最简型矩阵、主元。主元反映到线性方程组中又产生了主变量和自由变量的概念。
\begin{definition}[行阶梯形矩阵]
    一个矩阵是行阶梯型矩阵,当且仅当它满足:
    \begin{itemize}
        \item 若有零行,零行在最下方
        \item 对于非零行,从左到右第一个非零元称为主元,主元的列指标随着行指标的增大严格增大
    \end{itemize}
\end{definition}
\begin{definition}[行最简型矩阵]
    一个矩阵是行最简型矩阵,当且仅当它满足:
    \begin{itemize}
        \item 是行阶梯型矩阵
        \item 主元都是1
        \item 主元所在列的其它元素都是0
    \end{itemize}
\end{definition}
\begin{definition}[主变量]
    以主元为系数的便量称为主变量,其它变量称为自由变量
\end{definition}

\section{线性方程组的解的情形及判别准则}
根据一些例子不难猜想出线性方程组的解有三种情形:
\begin{enumerate}
    \item 当方程出现不相容时,无解
    \item 有唯一解
    \item 有无穷多解
\end{enumerate}
现予以说明:首先将对应的增广矩阵化为行阶梯型,设变量有$n$个,此时有$r$个非零行。
\begin{itemize}
    \item 若某行出现"$0=d(d\neq 0)$"时,显然无解
    \item 由主元的定义知有$r$个主元,故$r\leq n$\begin{itemize}
        \item 当$r=n$时,继续将行阶梯型化为行最简型,可直接得到此时有唯一解
        \item 当$r<n$时,存在自由变量,此时有无穷多解
    \end{itemize}
\end{itemize}

\begin{theorem}[线性方程组解的情况及判定准则]
    $n$元线性方程组的解的情形只有三种:
    \begin{itemize}
        \item 无解 $\Longleftrightarrow$ 出现"$0=d(d\neq0)$"
        \item 有唯一解 $\Longleftrightarrow$ 增广矩阵化行阶梯型非零行行数$r=n$
        \item 有无穷多解 $\Longleftrightarrow$ 增广矩阵化行阶梯型非零行行数$r<n$
    \end{itemize}
\end{theorem}
\begin{corollary}
    $n$元齐次线性方程组的解的情形有两种:
    \begin{itemize}
        \item 有唯一零解  $\Longleftrightarrow$ 系数矩阵化行阶梯型非零行行数$r=n$
        \item 有无穷多解 $\Longleftrightarrow$ 系数矩阵化行阶梯型非零行行数$r<n$
    \end{itemize}
\end{corollary}

\section{数域}
之前我们对线性方程组默认是系数对加减乘除都封闭,但是有些方程组在整数范围内不一定有解,例如:
\begin{equation*}
    \begin{cases}
        x+y=1\\
        x-y=0
    \end{cases}
\end{equation*}
这种情形对我们的求解造成了额外的困扰。因此,要使我们所做的初等变换是合理的,我们引入数域的概念。
\begin{definition}[数域]
    设集合$K$是复数集的一个子集,如果$K$满足:
    \begin{enumerate}
        \item $0,1\in K$
        \item $K$对加减乘除运算封闭
    \end{enumerate}
    则称$K$是一个数域
\end{definition}
\begin{corollary}
    根据定义容易验证最小的数域是有理数域,最大的数域是复数域
\end{corollary}

以后我们讨论的线性方程组都是数域$K$上的,即它的系数、常数项都属于$K$,并且在数域$K$里求它的解;
所讨论的矩阵,它的全部元素也属于$K$