\chapter{行列式}
为了直接从线性方程组的系数出发探究线性方程组的解的情况。我们先从解剖“麻雀”---2个未知数,2个线性方程的方程组开始。
对方程组:
\begin{equation*}
    \begin{cases}
        a_{11}x_1 + a_{12}x_2 = b_1\\
         a_{21}x_1 + a_{22}x_2 = b_2
     \end{cases}
\end{equation*}
\(a_{11},a_{21}\) 不全为0,不妨设$a_{11} \neq 0$,对其增广矩阵化行阶梯型有:
\begin{equation*}
    \begin{bmatrix}
        a_{11} & a_{12} & b_{1} \\
        0   &   a_{22}-a_{12}\frac{a_{21}}{a_{11}}  &   b_{2}-b_{1}\frac{a_{21}}{a_{11}}
    \end{bmatrix}
\end{equation*}
不难得出以下结论:
\begin{itemize}
    \item $a_{22}-a_{12}\frac{a_{21}}{a_{11}} \neq 0$ $\Leftrightarrow$ 原方程有唯一解
    \item $a_{22}-a_{12}\frac{a_{21}}{a_{11}} = 0$ $\Leftrightarrow$ 原方程无解或有无穷多解
\end{itemize}
由此我们给出二阶行列式的定义
\begin{definition}
    设$2$阶方阵
    \begin{equation*}
        A = \begin{bmatrix}
            a_{11} & a_{12}\\
            a_{21} & a_{22}
        \end{bmatrix}
    \end{equation*}
    ,则其行列式为
    \begin{equation*}
        det A=|A| = \begin{vmatrix}
            a_{11} & a_{12}\\
            a_{21} & a_{22}
        \end{vmatrix}
        = a_{11}a_{22} - a_{12}a_{21}
    \end{equation*}
\end{definition}
\begin{remark}
    $2$阶行列式有$2!$项,每一项的元素都取自不同行不同列,每一项的正负由行指标按顺序排列后,列指标排列的逆序对个数决定
\end{remark}
有上面的分析,容易得到以下定理:
\begin{theorem}
    对于两个方程的二元一次方程组:
    \begin{itemize}
        \item 有唯一解 $\Leftrightarrow$ 系数矩阵的行列式不等于零
        \item 无解或无穷多解 $\Leftrightarrow$ 系数矩阵的行列式等于零
    \end{itemize}
\end{theorem}

对于$n$个方程的$n$元线性方程组,能否用所谓的$n$阶行列式反应其解的情况呢?实际上是可以的。
\begin{remark}
    $n$阶行列式的定义方式并不自然,暂时也只能这样了。
\end{remark}

\section{n元排列}
由二阶行列式的定义与列指标排列的逆序对个数有关,为了给出$n$阶行列式的每一项的符号,我们需要先研究一个$n$元排列的奇偶性
\begin{definition}
    $n$个不同整数的一个全排列称为一个$n$元排列
\end{definition}
\begin{definition}
    一个$n$元排列中任意取两个数,若大的在前,小的在后,则这两个元素形成了一个逆序对。
\end{definition}
\begin{definition}
    对一个$n$元排列,若逆序对个数为奇数个,则称其为奇排列,反之称为偶排列。
\end{definition}
\begin{theorem}
    相邻两个元素交换改变排列的奇偶性
\end{theorem}
不难推出:
\begin{theorem}
    任意两个元素交换改变排列的奇偶性,即:
    \begin{equation*}
        (-1)^{p_{1}p_{2}\cdots p_{i}p_{j}}
    \end{equation*}
\end{theorem}

\section{n阶行列式的定义}
仿照$2$阶行列式,$n$阶行列式的每一项