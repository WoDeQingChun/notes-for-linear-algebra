\chapter{行列式}
为了直接从线性方程组的系数出发探究线性方程组的解的情况。我们先从解剖“麻雀”---2个未知数,2个线性方程的方程组开始。
对方程组:
\begin{equation*}
    \begin{cases}
        a_{11}x_1 + a_{12}x_2 = b_1\\
         a_{21}x_1 + a_{22}x_2 = b_2
     \end{cases}
\end{equation*}
\(a_{11},a_{21}\) 不全为0,不妨设$a_{11} \neq 0$,对其增广矩阵化行阶梯型有:
\begin{equation*}
    \begin{bmatrix}
        a_{11} & a_{12} & b_{1} \\
        0   &   a_{22}-a_{12}\frac{a_{21}}{a_{11}}  &   b_{2}-b_{1}\frac{a_{21}}{a_{11}}
    \end{bmatrix}
\end{equation*}
不难得出以下结论:
\begin{itemize}
    \item $a_{22}-a_{12}\frac{a_{21}}{a_{11}} \neq 0$ $\Leftrightarrow$ 原方程有唯一解
    \item $a_{22}-a_{12}\frac{a_{21}}{a_{11}} = 0$ $\Leftrightarrow$ 原方程无解或有无穷多解
\end{itemize}
由此我们给出二阶行列式的定义
\begin{definition}
    设$2$阶方阵
    \begin{equation*}
        A = \begin{bmatrix}
            a_{11} & a_{12}\\
            a_{21} & a_{22}
        \end{bmatrix}
    \end{equation*}
    ,则其行列式为
    \begin{equation*}
        det A=|A| = \begin{vmatrix}
            a_{11} & a_{12}\\
            a_{21} & a_{22}
        \end{vmatrix}
        = a_{11}a_{22} - a_{12}a_{21}
    \end{equation*}
\end{definition}
\begin{remark}
    $2$阶行列式有$2!$项,每一项的元素都取自不同行不同列,每一项的正负由行指标按顺序排列后,列指标排列的逆序对个数决定
\end{remark}
有上面的分析,容易得到以下定理:
\begin{theorem}
    对于两个方程的二元一次方程组:
    \begin{itemize}
        \item 有唯一解 $\Leftrightarrow$ 系数矩阵的行列式不等于零
        \item 无解或无穷多解 $\Leftrightarrow$ 系数矩阵的行列式等于零
    \end{itemize}
\end{theorem}

对于$n$个方程的$n$元线性方程组,能否用所谓的$n$阶行列式反应其解的情况呢?实际上是可以的。
\begin{remark}
    $n$阶行列式的定义方式并不自然,暂时也只能这样了。
\end{remark}

\section{n元排列}
由二阶行列式的定义与列指标排列的逆序对个数有关,为了给出$n$阶行列式的每一项的符号,我们需要先研究一个$n$元排列的奇偶性
\begin{definition}
    $n$个不同整数的一个全排列称为一个$n$元排列
\end{definition}
\begin{definition}
    一个$n$元排列中任意取两个数,若大的在前,小的在后,则这两个元素形成了一个逆序对。
\end{definition}
\begin{definition}
    对一个$n$元排列,若逆序对个数为奇数个,则称其为奇排列,反之称为偶排列。
\end{definition}
\begin{theorem}
    相邻两个元素交换改变排列的奇偶性
\end{theorem}
不难推出:
\begin{theorem}
    任意两个元素交换改变排列的奇偶性,即:
    \begin{equation*}
        (-1)^{\tau (p_{1}p_{2}\cdots p_{i-1} \bm{p_{i}} p_{i+1} \cdots p_{j-1} \bm{p_{j}} p_{j+1} \cdots p_{n})}
        = (-1) \times (-1)^{\tau (p_{1}p_{2}\cdots p_{i-1} \bm{p_{j}} p_{i+1} \cdots p_{j-1} \bm{p_{i}} p_{j+1} \cdots p_{n})}
    \end{equation*}
\end{theorem}

\section{n阶行列式的定义}
\subsection{行指标有序的定义方式}
仿照$2$阶行列式,$n$阶行列式的每一项都是$n$个不同行不同列的元素的乘积,共有$n!$项,
每一项的正负由行指标按顺序排列后,列指标排列的逆序对个数决定,因而对$n$阶行列式定义如下:
\begin{definition}
    设$n$阶方阵
    \begin{equation*}
        A = \begin{bmatrix}
            a_{11} & a_{12} & \cdots & a_{1n}\\
            a_{21} & a_{22} & \cdots & a_{2n}\\
            \vdots & \vdots &        & \vdots\\
            a_{n1} & a_{n2} & \cdots & a_{nn}
        \end{bmatrix}
    \end{equation*}
    定义其行列式
    \begin{equation*}
        det A = |A| = \sum_{j_{1}j_{2}\cdots j_{n}}^{} (-1)^{\tau (j_{1}j_{2}\cdots j_{n})}a_{1j_{1}}a_{2j_{2}}\cdots a_{nj_{n}}
    \end{equation*}
\end{definition}

\subsection{更一般的定义方式}
上面每一项的行指标都是有序排列的,但每一项的n个元素的乘积可以按任意次序相乘。即若将每一项的行指标混排,其符号该如何确定呢?\\
对$a_{1j_{1}}a_{2j_{2}}\cdots a_{nj_{n}}$这一项,假设经过$s$次对换变为排列$a_{k_{1}p_{1}}a_{k_{2}p_{2}}\cdots a_{k_{n}p_{n}}$,
则对于行指标排列的奇偶性和列指标排列的奇偶性分别有:
\begin{equation*}
    \begin{cases}
        (-1)^{s} (-1)^{\tau(12 \cdots n)} = (-1)^{\tau (k_{1}k_{2}\cdots k_{n})} \\
        (-1)^{s} (-1)^{\tau (p_{1}p_{2}\cdots p_{n})} = (-1)^{\tau (j_{1}j_{2}\cdots j_{n})}
    \end{cases}
\end{equation*}
故有
\begin{equation*}
    (-1)^{\tau (k_{1}k_{2}\cdots k_{n}) + \tau (p_{1}p_{2}\cdots p_{n})} = (-1)^{\tau (j_{1}j_{2}\cdots j_{n})}
\end{equation*}
和
\begin{equation*}
    (-1)^{\tau (k_{1}k_{2}\cdots k_{n}) + \tau (p_{1}p_{2}\cdots p_{n})} a_{k_{1}p_{1}}a_{k_{2}p_{2}}\cdots a_{k_{n}p_{n}} 
    =
    (-1)^{\tau (j_{1}j_{2}\cdots j_{n})} a_{1j_{1}}a_{2j_{2}}\cdots a_{nj_{n}}
\end{equation*}
由上面的分析,容易导出以下两条性质:
\begin{theorem}[行列式的一般定义]
    给定行指标的一个排列$k_{1}k_{2}\cdots k_{n}$, $n$级矩阵$A$的行列式
    \begin{equation*}
        |A| = \sum _{p_{1}p_{2}\cdots p_{n}} ^{} (-1)^{\tau (k_{1}k_{2}\cdots k_{n}) + \tau (p_{1}p_{2}\cdots p_{n})} a_{k_{1}p_{1}}a_{k_{2}p_{2}}\cdots a_{k_{n}p_{n}}
    \end{equation*}
\end{theorem}

若每一项按照列指标自然序排好位置,那么
\begin{theorem}
    \begin{equation*}
        |A| = \sum_{i_{1}i_{2}\cdots i_{n}}^{} (-1)^{\tau (i_{1}i_{2}\cdots i_{n})} 
        a_{i_{1}1}a_{i_{2}2}\cdots a_{i_{n}n}
    \end{equation*}
\end{theorem}
\begin{remark}
    上面这条定理其实已经说明行列式的行和列具有相同的地位
\end{remark}

\section{行列式的性质}
从探究系数矩阵$A$的初等变换会对它的行列式产生什么样的影响这个问题出发
我们容易探究出关于行列式的一系列性质:

\begin{enumerate}
    \item $|A| = |A^T|$这一条由定义可以直接导出,说明了对行成立的性质对列也成立,故下面只注明行的性质
    \item 一行的公因子可以提出
    \item 互换两行行列式变号
    \item 若某一行是两组数的和,则行列式可以拆成两个行列式的和,他们的这一样分别是上述两组数,其它行不变
    \item 有两行相同,则行列式的值为0
    \item 行列式中有两行成比例,则行列式的值为0
    \item 把一行的倍数加到另一行上,行列式的值不变
\end{enumerate}

以上性质全都可以由定义为出发点推导得来

\section{行列式按一行/列展开}
注意这样一句话:\emph{行列式的每一项都是不同行不同列的元素的乘积}