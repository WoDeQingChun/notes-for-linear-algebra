\chapter{行列式}
为了直接从线性方程组的系数出发探究线性方程组的解的情况。我们先从解剖“麻雀”---2个未知数,2个线性方程的方程组开始。
对方程组:
\begin{equation*}
    \begin{cases}
        a_{11}x_1 + a_{12}x_2 = b_1\\
         a_{21}x_1 + a_{22}x_2 = b_2
     \end{cases}
\end{equation*}
\(a_{11},a_{21}\) 不全为0,不妨设$a_{11} \neq 0$,对其增广矩阵化行阶梯型有:
\begin{equation*}
    \begin{bmatrix}
        a_{11} & a_{12} & b_{1} \\
        0   &   a_{22}-a_{12}\frac{a_{21}}{a_{11}}  &   b_{2}-b_{1}\frac{a_{21}}{a_{11}}
    \end{bmatrix}
\end{equation*}
不难得出以下结论:
\begin{itemize}
    \item $a_{22}-a_{12}\frac{a_{21}}{a_{11}} \neq 0$ $$\Longleftrightarrow$$ 原方程有唯一解
    \item $a_{22}-a_{12}\frac{a_{21}}{a_{11}} = 0$ $$\Longleftrightarrow$$ 原方程无解或有无穷多解
\end{itemize}
由此我们给出二阶行列式的定义
\begin{definition}
    设$2$阶方阵
    \begin{equation*}
        A = \begin{bmatrix}
            a_{11} & a_{12}\\
            a_{21} & a_{22}
        \end{bmatrix}
    \end{equation*}
    ,则其行列式为
    \begin{equation*}
        det A=|A| = \begin{vmatrix}
            a_{11} & a_{12}\\
            a_{21} & a_{22}
        \end{vmatrix}
        = a_{11}a_{22} - a_{12}a_{21}
    \end{equation*}
\end{definition}
\begin{remark}
    $2$阶行列式有$2!$项,每一项的元素都取自不同行不同列,每一项的正负由行指标按顺序排列后,列指标排列的逆序对个数决定
\end{remark}
有上面的分析,容易得到以下定理:
\begin{theorem}
    对于两个方程的二元一次方程组:
    \begin{itemize}
        \item 有唯一解 $\Leftrightarrow$ 系数矩阵的行列式不等于零
        \item 无解或无穷多解 $\Leftrightarrow$ 系数矩阵的行列式等于零
    \end{itemize}
\end{theorem}

对于$n$个方程的$n$元线性方程组,能否用所谓的$n$阶行列式反应其解的情况呢?实际上是可以的。
\begin{remark}
    $n$阶行列式的定义方式并不自然,暂时也只能这样了。
\end{remark}

\section{n元排列}
由二阶行列式的定义与列指标排列的逆序对个数有关,为了给出$n$阶行列式的每一项的符号,我们需要先研究一个$n$元排列的奇偶性
\begin{definition}
    $n$个不同整数的一个全排列称为一个$n$元排列
\end{definition}
\begin{definition}
    一个$n$元排列中任意取两个数,若大的在前,小的在后,则这两个元素形成了一个逆序对。
\end{definition}
\begin{definition}
    对一个$n$元排列,若逆序对个数为奇数个,则称其为奇排列,反之称为偶排列。
\end{definition}
\begin{theorem}
    相邻两个元素交换改变排列的奇偶性
\end{theorem}
不难推出:
\begin{theorem}
    任意两个元素交换改变排列的奇偶性,即:
    \begin{equation*}
        (-1)^{\tau (p_{1}p_{2}\cdots p_{i-1} \bm{p_{i}} p_{i+1} \cdots p_{j-1} \bm{p_{j}} p_{j+1} \cdots p_{n})}
        = (-1) \times (-1)^{\tau (p_{1}p_{2}\cdots p_{i-1} \bm{p_{j}} p_{i+1} \cdots p_{j-1} \bm{p_{i}} p_{j+1} \cdots p_{n})}
    \end{equation*}
\end{theorem}

\section{n阶行列式的定义}
\subsection{行指标有序的定义方式}
仿照$2$阶行列式,$n$阶行列式的每一项都是$n$个不同行不同列的元素的乘积,共有$n!$项,
每一项的正负由行指标按顺序排列后,列指标排列的逆序对个数决定,因而对$n$阶行列式定义如下:
\begin{definition}
    设$n$阶方阵
    \begin{equation*}
        A = \begin{bmatrix}
            a_{11} & a_{12} & \cdots & a_{1n}\\
            a_{21} & a_{22} & \cdots & a_{2n}\\
            \vdots & \vdots &        & \vdots\\
            a_{n1} & a_{n2} & \cdots & a_{nn}
        \end{bmatrix}
    \end{equation*}
    定义其行列式
    \begin{equation*}
        det A = |A| = \sum_{j_{1}j_{2}\cdots j_{n}}^{} (-1)^{\tau (j_{1}j_{2}\cdots j_{n})}a_{1j_{1}}a_{2j_{2}}\cdots a_{nj_{n}}
    \end{equation*}
\end{definition}

\subsection{更一般的定义方式}
上面每一项的行指标都是有序排列的,但每一项的n个元素的乘积可以按任意次序相乘。即若将每一项的行指标混排,其符号该如何确定呢?\\
对$a_{1j_{1}}a_{2j_{2}}\cdots a_{nj_{n}}$这一项,假设经过$s$次对换变为排列$a_{k_{1}p_{1}}a_{k_{2}p_{2}}\cdots a_{k_{n}p_{n}}$,
则对于行指标排列的奇偶性和列指标排列的奇偶性分别有:
\begin{equation*}
    \begin{cases}
        (-1)^{s} (-1)^{\tau(12 \cdots n)} = (-1)^{\tau (k_{1}k_{2}\cdots k_{n})} \\
        (-1)^{s} (-1)^{\tau (p_{1}p_{2}\cdots p_{n})} = (-1)^{\tau (j_{1}j_{2}\cdots j_{n})}
    \end{cases}
\end{equation*}
故有
\begin{equation*}
    (-1)^{\tau (k_{1}k_{2}\cdots k_{n}) + \tau (p_{1}p_{2}\cdots p_{n})} = (-1)^{\tau (j_{1}j_{2}\cdots j_{n})}
\end{equation*}
和
\begin{equation*}
    (-1)^{\tau (k_{1}k_{2}\cdots k_{n}) + \tau (p_{1}p_{2}\cdots p_{n})} a_{k_{1}p_{1}}a_{k_{2}p_{2}}\cdots a_{k_{n}p_{n}} 
    =
    (-1)^{\tau (j_{1}j_{2}\cdots j_{n})} a_{1j_{1}}a_{2j_{2}}\cdots a_{nj_{n}}
\end{equation*}
由上面的分析,容易导出以下两条性质:
\begin{theorem}[行列式的一般定义]
    给定行指标的一个排列$k_{1}k_{2}\cdots k_{n}$, $n$级矩阵$A$的行列式
    \begin{equation*}
        |A| = \sum _{p_{1}p_{2}\cdots p_{n}} ^{} (-1)^{\tau (k_{1}k_{2}\cdots k_{n}) + \tau (p_{1}p_{2}\cdots p_{n})} a_{k_{1}p_{1}}a_{k_{2}p_{2}}\cdots a_{k_{n}p_{n}}
    \end{equation*}
\end{theorem}

若每一项按照列指标自然序排好位置,那么
\begin{theorem}
    \begin{equation*}
        |A| = \sum_{i_{1}i_{2}\cdots i_{n}}^{} (-1)^{\tau (i_{1}i_{2}\cdots i_{n})} 
        a_{i_{1}1}a_{i_{2}2}\cdots a_{i_{n}n}
    \end{equation*}
\end{theorem}
\begin{remark}
    上面这条定理其实已经说明行列式的行和列具有相同的地位
\end{remark}

\section{行列式的性质}
从探究系数矩阵$A$的初等变换会对它的行列式产生什么样的影响这个问题出发
我们容易探究出关于行列式的一系列性质:

\begin{enumerate}
    \item $|A| = |A^T|$这一条由定义可以直接导出,说明了对行成立的性质对列也成立,故下面只注明行的性质
    \item 一行的公因子可以提出
    \item 互换两行行列式变号
    \item 若某一行是两组数的和,则行列式可以拆成两个行列式的和,他们的这一样分别是上述两组数,其它行不变
    \item 有两行相同,则行列式的值为0
    \item 行列式中有两行成比例,则行列式的值为0
    \item 把一行的倍数加到另一行上,行列式的值不变
\end{enumerate}

以上性质全都可以由定义为出发点推导得来

\section{行列式按一行/列展开}
注意这样一句话:\emph{行列式的每一项都是不同行不同列的元素的乘积},因而在取定一个元素$a_{ij}$后,d第$i$行,第$j$列的元素都不能取了。

由于$n$阶行列式可以按一行一直分解至$n$个行列式的和,不难得出行列式按一行(列)展开的定理

先铺垫如下概念:
\begin{definition}[(代数)余子式]
    $n$阶行列式$|A|$中,取定一个元素$a_{ij}$,则划去该元素所在的第$i$行,第$j$列,剩下的元素按原来次序组成的$n-1$阶行列式称为元素
    $a_{ij}$的余子式,记为$M_{ij}$,$(-1)^{i+j}M_{ij}$称为元素$a_{ij}$的代数余子式,记为$A_{ij}$
\end{definition}

\begin{theorem}
行列式的一行(列)展开
对$n$阶行列式$|A|$,
    \begin{itemize}
        \item 按第$i$行展开有:\begin{equation*}
            |A| = \sum_{j=1}^{n}a_{ij}A_{ij}
        \end{equation*}
        \item 按第$j$列展开有:\begin{equation*}
            |A| = \sum_{i=1}^{n}a_{ij}A_{ij}
        \end{equation*}
    \end{itemize}
\end{theorem}

容易从行列式的分解性质和交换行的性质得到,先给出按一行展开的形式化的证明
\begin{proof}
对$n$阶行列式$|A|$,取定第$i$行,并把每一项的元素$a_{ij}$放到第一个位置,
其余$n-1$个元素按自然序排好位置,根据列指标$j = 1, 2, \cdots, n$将$|A|$的$n!$项分成$n$组,则有:
\begin{equation*}
    \begin{split}
        |A| & = \sum_{j k_{1} \cdots k_{i-1} k_{i+1} \cdots k_{n}}^{} (-1)^{\tau (i 1 \cdots, i-1,i+1,\cdots n) + \tau(jk_{1} \cdots k_{i-1}k_{i+1} \cdots k_{n})} 
        a_{ij}a_{1k_{1}} \cdots a_{i-1,k_{i-1}} a_{i+1,k_{i+1}} \cdots a_{nk_{n}}\\
            & =  \sum_{j k_{1} \cdots k_{i-1} k_{i+1} \cdots k_{n}}^{} (-1)^{i-1 + j-1 + \tau(k_{1} \cdots k_{i-1}k_{i+1} \cdots k_{n})}
        a_{ij}a_{1k_{1}} \cdots a_{i-1,k_{i-1}} a_{i+1,k_{i+1}} \cdots a_{nk_{n}}\\
            & = \sum_{j k_{1} \cdots k_{i-1} k_{i+1} \cdots k_{n}}^{} (-1)^{i+j}a_{ij} \cdot (-1)^{\tau(k_{1} \cdots k_{i-1}k_{i+1} \cdots k_{n})}a_{1k_{1}} \cdots a_{i-1,k_{i-1}} a_{i+1,k_{i+1}} \cdots a_{nk_{n}}\\
            & = \sum_{j=1}^{n} (-1)^{i+j}a_{ij} \Biggl[ \sum_{k_{1} \cdots k_{i-1} k_{i+1} \cdots k_{n}}^{} (-1)^{\tau(k_{1} \cdots k_{i-1}k_{i+1} \cdots k_{n})}a_{1k_{1}} \cdots a_{i-1,k_{i-1}} a_{i+1,k_{i+1}} \cdots a_{nk_{n}} \Biggr]\\
            & = \sum_{j=1}^{n} (-1)^{i+j} a_{ij} M_{ij}(\text{根据定义})\\
            & = \sum_{j=1}^{n} a_{ij} A_{ij}
    \end{split}
\end{equation*}
    证毕
\end{proof}

\section{行列式按$k$行/列展开}
\begin{definition}[$k$阶子式]
    对$n$级矩阵$A$,任取
    $k$行$i_1,\cdots, i_k$,其中$1 \leq i_1 < \cdots < i_k \leq n$,
    $k$列$j_1,\cdots, j_k$,其中$1 \leq j_1 < \cdots < j_k \leq n$,
    这$k$行$k$列交叉处按原来的次序形成的行列式称为$A$的$k$阶子式,记为
    \begin{equation*}
        A
        \begin{pmatrix}
            i_1 & \cdots & i_k\\
            j_1 & \cdots & j_k
        \end{pmatrix}
    \end{equation*}
\end{definition}

\begin{definition}[(代数)余子式]
    令$\{ i_{1}^{'},\cdots, i_{n-k}^{'} \} = \{ 1,2,\cdots, n \} \backslash \{ i_1,\cdots, i_k \}$,其中$1 \leq i_{1}^{'} \leq \cdots \leq i_{n-k}^{'}$;
    $\{ j_{1}^{'},\cdots, j_{n-k}^{'} \} = \{ 1,2,\cdots, n \} \backslash \{ j_1,\cdots, j_k \}$,其中$1 \leq j_{1}^{'} \leq \cdots \leq j_{n-k}^{'}$,
    则称
    \begin{equation*}
        A
        \begin{pmatrix}
            i_{1}^{'} & \cdots & i_{n-k}^{'}\\
            j_{1}^{'} & \cdots & j_{n-k}^{'}\\
        \end{pmatrix}
    \end{equation*}
    为上述$k$阶子式的余子式
    称
    \begin{equation*}
        (-1)^{(i_1 + \cdots + i_k) + (j_1 + \cdots + j_k)}A
        \begin{pmatrix}
            i_{1}^{'} & \cdots & i_{n-k}^{'}\\
            j_{1}^{'} & \cdots & j_{n-k}^{'}\\
        \end{pmatrix}
    \end{equation*}
    为上述$k$阶子式的代数余子式
\end{definition}

\begin{theorem}[Laplace定理]
    取定行列式$|A|$的$k$行,则这$k$行形成的$k$阶子式与其对应的代数余子式乘积之和就是行列式的值,即
    \begin{equation*}
        |A| = \sum_{1 \leq j_1 < \cdots < j_k \leq n}^{}A
        \begin{pmatrix}
            i_{1} & \cdots & i_{k}\\
            j_{1} & \cdots & j_{k}\\
        \end{pmatrix}
        (-1)^{(i_1 + \cdots + i_k) + (j_1 + \cdots + j_k)}A
        \begin{pmatrix}
            i_{1}^{'} & \cdots & i_{n-k}^{'}\\
            j_{1}^{'} & \cdots & j_{n-k}^{'}\\
        \end{pmatrix}
    \end{equation*}
\end{theorem}

\begin{proof}
    取定$|A|$的$k$行,将这$k$行形成的$\mathbf{C}_n^k$个$k$阶子式分组,并把这$k$行的$k$
    个元素排在最前面,则:
    \begin{equation*}
        |A| = \sum_{u_1 \cdots u_k v_1 \cdots v_{n-k}}^{} (-1)^{i_1 + \cdots + i_k - \frac{k(k+1)}{2} \tau(u_1 \cdots u_k v_1 \cdots v_k)} 
        a_{i_1 u_1} \cdots a_{i_k u_k} a_{i_{1}^{'} v_1} \cdots a_{i_{n-k}^{'} v_{n-k}}
    \end{equation*}
    其中$u_1 \cdots u_k$是$j_1,\cdots, j_k$的一个$k$元排列;
    $v_1 \cdots v_{n-k}$是$j_{1}^{'},\cdots, j_{n-k}^{'}$的一个$n-k$元排列;
    分组思路为:
    \begin{enumerate}
        \item 先从$n$列中任取$k$列:$j_1,\cdots, j_k$
        \item $u_1 \cdots u_k$是列指标$j_1,\cdots, j_k$的一个全排列
        \item $v_1 \cdots v_{n-k}$是剩下的列指标$j_{1}^{'},\cdots, j_{n-k}^{'}$的一个全排列
    \end{enumerate}
    另外,假设排列$u_1 \cdots u_k$经过$s$次对换变为$j_1 \cdots j_k$则:
     
     \begin{equation*}
         (-1)^{\tau(u_1 \cdots u_k)} = (-1)^{s} (-1)^{\tau(j_1,\cdots, j_k)} = (-1)^{s}
     \end{equation*}
    从而:
    \begin{equation*}
        \begin{split}
            &(-1)^{\tau(u_1 \cdots u_k v_1 \cdots v_{n-k})}\\ 
            & = (-1)^{s} (-1)^{\tau(j_1 \cdots j_k v_1 \cdots v_{n-k})}\\
            & = (-1)^{\tau(u_1 \cdots u_k)} (-1)^{j_1 + \cdots + j_k - \frac{k(k+1)}{2} + \tau(v_1 \cdots v_{n-k})}\\
            & = (-1)^{j_1 + \cdots + j_k - \frac{k(k+1)}{2}} (-1)^{\tau(u_1 \cdots u_k) + \tau(v_1 \cdots v_{n-k})}\\
            \end{split}
    \end{equation*}
    于是有:
    \begin{equation*}
        \begin{split}
            |A| &= \sum_{u_1 \cdots u_k v_1 \cdots v_{n-k}}^{} (-1)^{\tau(i_1 \cdots i_k i_{1}^{'} \cdots i_{n-k}^{'}) + \tau(u_1 \cdots u_k v_1 \cdots v_k)} 
        a_{i_1 u_1} \cdots a_{i_k u_k} a_{i_{1}^{'} v_1} \cdots a_{i_{n-k}^{'} v_{n-k}} \\
                &= \sum_{u_1 \cdots u_k v_1 \cdots v_{n-k}}^{} (-1)^{i_1 + \cdots + i_k - \frac{k(k+1)}{2} + \tau(u_1 \cdots u_k v_1 \cdots v_k)} 
        a_{i_1 u_1} \cdots a_{i_k u_k} a_{i_{1}^{'} v_1} \cdots a_{i_{n-k}^{'} v_{n-k}} \\
                &= \sum_{1 \leq j_1 < \cdots < j_k} ^{} \sum_{u_1 \cdots u_k}^{} \sum_{v_1 \cdots v_{n-k}}
            (-1)^{i_1 + \cdots + i_k - \frac{k(k+1)}{2} + j_1 + \cdots + j_k - \frac{k(k+1)}{2}} (-1)^{\tau(u_1 \cdots u_k) + \tau(v_1 \cdots v_{n-k})} \cdot\\
                &a_{i_1 u_1} \cdots a_{i_k u_k} a_{i_{1}^{'} v_1} \cdots a_{i_{n-k}^{'} v_{n-k}} \\
                &= \sum_{1 \leq j_1 < \cdots < j_k} ^{} (-1)^{(i_1 + \cdots + i_k) + (j_1 + \cdots + j_k)}
                \sum_{u_1 \cdots u_k}^{} (-1)^{\tau(u_1 \cdots u_k)} a_{i_1 u_1} \cdots a_{i_k u_k} \sum_{v_1 \cdots v_{n-k}} (-1)^{\tau(v_1 \cdots v_k)}a_{i_{1}^{'} v_1} \cdots a_{i_{n-k}^{'} v_{n-k}}\\
                &= \sum_{1 \leq j_1 < \cdots < j_k \leq n}^{}A
                \begin{pmatrix}
                    i_{1} & \cdots & i_{k}\\
                    j_{1} & \cdots & j_{k}\\
                \end{pmatrix}
                (-1)^{(i_1 + \cdots + i_k) + (j_1 + \cdots + j_k)}A
                \begin{pmatrix}
                    i_{1}^{'} & \cdots & i_{n-k}^{'}\\
                    j_{1}^{'} & \cdots & j_{n-k}^{'}\\
                \end{pmatrix}
                \text{(根据定义)}
        \end{split}
    \end{equation*}
\end{proof}

\section{范德蒙(Vandermonde)行列式}
数学归纳法易证:
\begin{equation*}
    \begin{vmatrix}
        1 & 1 & \cdots & 1  \\
        x_1 & x_2 & \cdots & x_n \\
        x_1^2 & x_2^2 &\cdots & x_n^2 \\
        \vdots & \vdots & & \vdots \\
        x_1^{n-1} & x_2^{n-1} &\cdots & x_n^{n-1} \\
    \end{vmatrix}
    = \prod_{1 \leq j < i \leq n}^{}(x_i - x_j)
\end{equation*}

\section{Cramer法则及其补充}

\begin{theorem}[$n$个方程的$n$元线性方程组解的行列式判定准则]
    对含$n$个方程的$n$元线性方程组,设其系数矩阵为$A$,增广矩阵为$\widetilde{A}$,
    对应的行阶梯型矩阵分别为$J$,$\widetilde{J}$,则$|J| = \lambda  |A|(\lambda \neq 0)$.且有如下判定定理:
    \begin{itemize}
        \item 无解 $\Longleftrightarrow$ $\widetilde{J}$中出现"$0=d(d \neq 0)$" $\mathop{\Longleftrightarrow}\limits^{\text{左箭头可由}}_{\text{反证法得到}}$ $J$中的有零行 $\Longleftrightarrow$ $|J| = 0$ $\Longleftrightarrow$ $|A| = 0$
        \item 有唯一解 $\Longleftrightarrow$ 增广矩阵化行阶梯型非零行行数$r=n$ $\Longleftrightarrow$ $J$中有$n$个主元 $\Longleftrightarrow$ $|J| \neq 0$ $\Longleftrightarrow$ $|A| \neq 0$
        \item 有无穷多解 $\Longleftrightarrow$ 增广矩阵化行阶梯型非零行行数$r<n$ $\Longleftrightarrow$ $J$中的有零行 $\Longleftrightarrow$ $|J| = 0$ $\Longleftrightarrow$ $|A| = 0$
    \end{itemize}
\end{theorem}
\begin{corollary}
    $n$元齐次线性方程组的解的情形有两种:
    \begin{itemize}
        \item 有唯一零解  $\Longleftrightarrow$ 系数矩阵化行阶梯型非零行行数$r=n$ $\Longleftrightarrow$ $|A| \neq 0$
        \item 有非零解/有无穷多解 $\Longleftrightarrow$ 系数矩阵化行阶梯型非零行行数$r<n$ $\Longleftrightarrow$ $|A| = 0$
    \end{itemize}
\end{corollary}

TODO: Cramer法则后面会利用矩阵证明
